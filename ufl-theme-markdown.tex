% Options for packages loaded elsewhere
\PassOptionsToPackage{unicode}{hyperref}
\PassOptionsToPackage{hyphens}{url}
%
\documentclass[
  ignorenonframetext,
  aspectratio=169,
]{beamer}
\usepackage{pgfpages}
\setbeamertemplate{caption}[numbered]
\setbeamertemplate{caption label separator}{: }
\setbeamercolor{caption name}{fg=normal text.fg}
\beamertemplatenavigationsymbolsempty
% Prevent slide breaks in the middle of a paragraph
\widowpenalties 1 10000
\raggedbottom
\usepackage{amsmath,amssymb}
\usepackage{lmodern}
\usepackage{iftex}
\ifPDFTeX
  \usepackage[T1]{fontenc}
  \usepackage[utf8]{inputenc}
  \usepackage{textcomp} % provide euro and other symbols
\else % if luatex or xetex
  \usepackage{unicode-math}
  \defaultfontfeatures{Scale=MatchLowercase}
  \defaultfontfeatures[\rmfamily]{Ligatures=TeX,Scale=1}
\fi
\usetheme[]{ufl}
% Use upquote if available, for straight quotes in verbatim environments
\IfFileExists{upquote.sty}{\usepackage{upquote}}{}
\IfFileExists{microtype.sty}{% use microtype if available
  \usepackage[]{microtype}
  \UseMicrotypeSet[protrusion]{basicmath} % disable protrusion for tt fonts
}{}
\makeatletter
\@ifundefined{KOMAClassName}{% if non-KOMA class
  \IfFileExists{parskip.sty}{%
    \usepackage{parskip}
  }{% else
    \setlength{\parindent}{0pt}
    \setlength{\parskip}{6pt plus 2pt minus 1pt}}
}{% if KOMA class
  \KOMAoptions{parskip=half}}
\makeatother
\usepackage{xcolor}
\IfFileExists{xurl.sty}{\usepackage{xurl}}{} % add URL line breaks if available
\IfFileExists{bookmark.sty}{\usepackage{bookmark}}{\usepackage{hyperref}}
\hypersetup{
  pdftitle={Beamer themes for the University of Florida},
  pdfauthor={Jason Cory Brunson},
  hidelinks,
  pdfcreator={LaTeX via pandoc}}
\urlstyle{same} % disable monospaced font for URLs
\newif\ifbibliography
\usepackage{color}
\usepackage{fancyvrb}
\newcommand{\VerbBar}{|}
\newcommand{\VERB}{\Verb[commandchars=\\\{\}]}
\DefineVerbatimEnvironment{Highlighting}{Verbatim}{commandchars=\\\{\}}
% Add ',fontsize=\small' for more characters per line
\newenvironment{Shaded}{}{}
\newcommand{\AlertTok}[1]{\textcolor[rgb]{1.00,0.00,0.00}{\textbf{#1}}}
\newcommand{\AnnotationTok}[1]{\textcolor[rgb]{0.38,0.63,0.69}{\textbf{\textit{#1}}}}
\newcommand{\AttributeTok}[1]{\textcolor[rgb]{0.49,0.56,0.16}{#1}}
\newcommand{\BaseNTok}[1]{\textcolor[rgb]{0.25,0.63,0.44}{#1}}
\newcommand{\BuiltInTok}[1]{#1}
\newcommand{\CharTok}[1]{\textcolor[rgb]{0.25,0.44,0.63}{#1}}
\newcommand{\CommentTok}[1]{\textcolor[rgb]{0.38,0.63,0.69}{\textit{#1}}}
\newcommand{\CommentVarTok}[1]{\textcolor[rgb]{0.38,0.63,0.69}{\textbf{\textit{#1}}}}
\newcommand{\ConstantTok}[1]{\textcolor[rgb]{0.53,0.00,0.00}{#1}}
\newcommand{\ControlFlowTok}[1]{\textcolor[rgb]{0.00,0.44,0.13}{\textbf{#1}}}
\newcommand{\DataTypeTok}[1]{\textcolor[rgb]{0.56,0.13,0.00}{#1}}
\newcommand{\DecValTok}[1]{\textcolor[rgb]{0.25,0.63,0.44}{#1}}
\newcommand{\DocumentationTok}[1]{\textcolor[rgb]{0.73,0.13,0.13}{\textit{#1}}}
\newcommand{\ErrorTok}[1]{\textcolor[rgb]{1.00,0.00,0.00}{\textbf{#1}}}
\newcommand{\ExtensionTok}[1]{#1}
\newcommand{\FloatTok}[1]{\textcolor[rgb]{0.25,0.63,0.44}{#1}}
\newcommand{\FunctionTok}[1]{\textcolor[rgb]{0.02,0.16,0.49}{#1}}
\newcommand{\ImportTok}[1]{#1}
\newcommand{\InformationTok}[1]{\textcolor[rgb]{0.38,0.63,0.69}{\textbf{\textit{#1}}}}
\newcommand{\KeywordTok}[1]{\textcolor[rgb]{0.00,0.44,0.13}{\textbf{#1}}}
\newcommand{\NormalTok}[1]{#1}
\newcommand{\OperatorTok}[1]{\textcolor[rgb]{0.40,0.40,0.40}{#1}}
\newcommand{\OtherTok}[1]{\textcolor[rgb]{0.00,0.44,0.13}{#1}}
\newcommand{\PreprocessorTok}[1]{\textcolor[rgb]{0.74,0.48,0.00}{#1}}
\newcommand{\RegionMarkerTok}[1]{#1}
\newcommand{\SpecialCharTok}[1]{\textcolor[rgb]{0.25,0.44,0.63}{#1}}
\newcommand{\SpecialStringTok}[1]{\textcolor[rgb]{0.73,0.40,0.53}{#1}}
\newcommand{\StringTok}[1]{\textcolor[rgb]{0.25,0.44,0.63}{#1}}
\newcommand{\VariableTok}[1]{\textcolor[rgb]{0.10,0.09,0.49}{#1}}
\newcommand{\VerbatimStringTok}[1]{\textcolor[rgb]{0.25,0.44,0.63}{#1}}
\newcommand{\WarningTok}[1]{\textcolor[rgb]{0.38,0.63,0.69}{\textbf{\textit{#1}}}}
\setlength{\emergencystretch}{3em} % prevent overfull lines
\providecommand{\tightlist}{%
  \setlength{\itemsep}{0pt}\setlength{\parskip}{0pt}}
\setcounter{secnumdepth}{-\maxdimen} % remove section numbering
% formatting environments
\newcommand{\columnsbegin}{\begin{columns}}
\newcommand{\columnsend}{\end{columns}}

% general text block environments
\newcommand{\blockbegin}[1]{\begin{block}{#1}}
\newcommand{\blockend}{\end{block}}
\newcommand{\alertblockbegin}[1]{\begin{alertblock}{#1}}
\newcommand{\alertblockend}{\end{alertblock}}

% mathematical environments
\newcommand{\definitionbegin}[1]{\begin{definition}{#1}}
\newcommand{\definitionend}{\end{definition}}
\newcommand{\examplebegin}[1]{\begin{example}{#1}}
\newcommand{\exampleend}{\end{example}}
\newcommand{\theorembegin}[1]{\begin{theorem}{#1}}
\newcommand{\theoremend}{\end{theorem}}
\newcommand{\proofbegin}[1]{\begin{proof}{#1}}
\newcommand{\proofend}{\end{proof}}
\newcommand{\lemmabegin}[1]{\begin{lemma}{#1}}
\newcommand{\lemmaend}{\end{lemma}}
\newcommand{\corollarybegin}[1]{\begin{corollary}{#1}}
\newcommand{\corollaryend}{\end{corollary}}
\ifLuaTeX
  \usepackage{selnolig}  % disable illegal ligatures
\fi

\title{Beamer themes for the University of Florida}
\author{Jason Cory Brunson}
\date{\today}
\institute{Laboratory for Systems Medicine, University of Florida}

\begin{document}
\frame{\titlepage}

\begin{frame}[allowframebreaks]
  \tableofcontents[hideallsubsections]
\end{frame}
\begin{frame}[fragile]{Markdown + Pandoc}
\protect\hypertarget{markdown-pandoc}{}
The \texttt{ufl} \textbf{Beamer theme} will be described in more detail
in \texttt{ufl-theme-example.pdf}, generated from
\texttt{ufl-theme-example.tex} using all four themes. (Currently, only
the color and font themes are drafted.) This slideshow is generated
instead from the \textbf{Markdown} file \texttt{ufl-theme-markdown.md}
using the document conversion program
\href{https://pandoc.org/}{\textbf{Pandoc}}.

Note that Markdown syntax is not standardized, so much so that i decided
not to link to any specific introduction or guide in the previous
paragraph. Features may differ between the \textbf{Pandoc Beamer}
converter and other engines, so beware that the syntax used here will
not necessarily work in, say, GitHub.
\end{frame}

\hypertarget{pandoc-beamer}{%
\section{Pandoc Beamer}\label{pandoc-beamer}}

\begin{frame}[fragile]{Markdown to \LaTeX~in Pandoc}
\protect\hypertarget{markdown-to-in-pandoc}{}
Render this document directly into a PDF using Pandoc from the command
line:

\begin{verbatim}
pandoc ufl-theme-markdown.md \
  -t beamer \
  --include-in-header=environment-shortcuts.tex \
  --toc \
  -o ufl-theme-markdown.pdf
\end{verbatim}

Change the output format to \TeX~by substituting the final option:

\begin{verbatim}
  -o ufl-theme-markdown.tex \
  -s
\end{verbatim}

(This will produce an intermediary standalone \textbf{\LaTeX~source
file}.)
\end{frame}

\begin{frame}[fragile]{Metadata}
\protect\hypertarget{metadata}{}
The \textbf{\href{https://yaml.org/}{YAML} front matter} at the top of
\texttt{ufl-theme-markdown.md} contains several metadata, including the
title, author, institution, and date. This document also sets the
following variables:

\begin{Shaded}
\begin{Highlighting}[]
\FunctionTok{aspectratio}\KeywordTok{:}\AttributeTok{ }\DecValTok{169}
\FunctionTok{section{-}titles}\KeywordTok{:}\AttributeTok{ }\CharTok{false}
\FunctionTok{theme}\KeywordTok{:}\AttributeTok{ ufl}
\end{Highlighting}
\end{Shaded}

The \texttt{theme} variable calls the \texttt{ufl} Beamer theme. The
\texttt{themeoptions} variable receives options that are passed to the
theme.

Find a list of Beamer-specific variables that can be set in the front
matter at the Pandoc User's Guide:

\begin{verbatim}
https://pandoc.org/MANUAL.html#variables-for-beamer-slides
\end{verbatim}
\end{frame}

\begin{frame}[fragile]{Environment shortcuts}
\protect\hypertarget{environment-shortcuts}{}
The Pandoc command above included one atypical option:

\begin{verbatim}
  --include-in-header=environment-shortcuts.tex
\end{verbatim}

This adds the contents of \texttt{environment-shortcuts.tex} to the
\LaTeX~preamble when rendering. The file contains several definitions
like this:

\begin{Shaded}
\begin{Highlighting}[]
\FunctionTok{\textbackslash{}newcommand}\NormalTok{\{}\ExtensionTok{\textbackslash{}blockbegin}\NormalTok{\}[1]\{}\KeywordTok{\textbackslash{}begin}\NormalTok{\{}\ExtensionTok{block}\NormalTok{\}\{\#1\}\}}
\FunctionTok{\textbackslash{}newcommand}\NormalTok{\{}\ExtensionTok{\textbackslash{}blockend}\NormalTok{\}\{}\KeywordTok{\textbackslash{}end}\NormalTok{\{}\ExtensionTok{block}\NormalTok{\}\}}
\end{Highlighting}
\end{Shaded}

These allow to use \LaTeX~environments without the use of
\texttt{\textbackslash{}begin\{\}} and \texttt{\textbackslash{}end\{\}},
thereby enabling Pandoc to render Markdown syntax within these
environments.
\end{frame}

\hypertarget{formatting-text}{%
\section{Formatting text}\label{formatting-text}}

\begin{frame}[fragile]{Blocks}
\protect\hypertarget{blocks}{}
\blockbegin{Text Block}

This text block is rendered using \verb|\blockbegin{Block Title}| and
\verb|\blockend|. Since Pandoc interprets single carriage returns as
spaces, two are required to separate these commands from the text they
contain.

Block titles are rendered in \primarytype{Cantarell extra bold}, a free
alternative to the Gothic bold used for campus, college, and school
woodmarks.

\blockend

\theorembegin

Likewise, a theorem can include \textbf{bold}, \emph{emphatic}, and
\texttt{fixed-width} font rendered from Markdown syntax. Note that
\verb|\emph{}| is required for emphatic text when \emph{italics} are
ignored in a theorem block, and notice the difference between
\texttt{fixed-width} and \verb|inline verbatim code|, which can still be
rendered using \verb|\verb|.

\theoremend
\end{frame}

\begin{frame}[fragile]{Lists}
\protect\hypertarget{lists}{}
Enumerated, itemized, and nested lists can be intuitively typed:

\begin{Shaded}
\begin{Highlighting}[]
\SpecialStringTok{1. }\NormalTok{This is a first item.}
\SpecialStringTok{2. }\NormalTok{This is a second, but it has}
\SpecialStringTok{    {-} }\NormalTok{not one,}
\SpecialStringTok{    {-} }\NormalTok{not two, but}
\SpecialStringTok{    {-} }\NormalTok{three sub{-}items.}
\end{Highlighting}
\end{Shaded}

The code above renders as follows:

\begin{enumerate}
\tightlist
\item
  This is a first item.
\item
  This is a second, but it has

  \begin{itemize}
  \tightlist
  \item
    not one,
  \item
    not two, but
  \item
    three sub-items.
  \end{itemize}
\end{enumerate}
\end{frame}

\begin{frame}[fragile]{Code blocks}
\protect\hypertarget{code-blocks}{}
The code blocks on the preceding slides use either 4-space indentation
(for command line code) or triple--back ticks with syntax highlighting
specific to the language on display (\texttt{yaml}, \texttt{tex}, and
\texttt{markdown}).

These and many other shorthands are documented in the
\href{https://pandoc.org/MANUAL.html}{Pandoc User's Guide}.
\end{frame}

\hypertarget{thanks}{%
\section{Thanks}\label{thanks}}

\begin{frame}{Acknowledgments}
\protect\hypertarget{acknowledgments}{}
Since i didn't recognize them in previous documents, i'll thank here the
designers of \TeX~and \LaTeX, as well as those of Markdown and Pandoc,
for making this exceedingly easy, while necessarily limited, plain
text--to--PDF workflow possible.

As in the other example documents, suggestions are very welcome! Email
Cory
(\href{mailto:jason.brunson@medicine.ufl.edu}{\nolinkurl{jason.brunson@medicine.ufl.edu}}),
raise an issue, or submit a pull request (on a new branch).
\end{frame}

\end{document}
