\documentclass{beamer}

\usepackage[utf8]{inputenc}
\usepackage[T1]{fontenc}

\useinnertheme{default}
\useoutertheme[footline=authorinstitute,subsection=true]{miniframes}
%\useoutertheme{sidebar}
\usecolortheme{ufl}
\usefonttheme{ufl}

\setbeamercolor*{date}{fg=ufl blue}

\title[Beamer colors and fonts for UF]{Beamer color and font themes\\ for the University of Florida}
\subtitle{part of a suite of themes}
\author[Cory Brunson]{Jason Cory Brunson, PhD}
\institute[University of Florida]{Laboratory for Systems Medicine\\ Division of Pulmonary, Critical Care, and Sleep Medicine\\ University of Florida}
\date{\today}

\newtheorem{remark}{Remark}

\begin{document}


\begin{frame}
\titlepage
\end{frame}


\begin{frame}{Table of contents}
\tableofcontents
\end{frame}


\section{Theme defaults}


\begin{frame}[fragile]{Default font family}

%The default Helvetica font family is provided by the {\ttfamily helvet} package.
%It is equivalent to Arial, which is used in the PowerPoint templates (see the \hyperlink{slide:acknowledgments}{Acknowledgments slide}).

The default font family is set to sans-serif, and the sans-serif default Cantarell is provided by \href{https://www.ctan.org/tex-archive/fonts/cantarell}{the {\ttfamily cantarell} package}. This is a free alternative to \href{http://identity.ufl.edu/typography/}{Gentona}, one of two official typefaces of the UF brand.

Block titles are rendered in {\cantarelleb Cantarell extra bold}, and two shortcuts are available to render any text in this style:
\begin{itemize}
\item \verb|\cantarelleb| (with no argument)
\item \verb|\primarytype{}| (with a text argument)
\end{itemize}

\end{frame}


\begin{frame}[fragile]{Palette colors}

The palette colors used by outer themes interpolate between UF blue (main) or UF orange (sidebar) and white:

\vfill

\begin{columns}
\begin{column}{.5\textwidth}

\centering
\verb|palette|
\vspace{1ex}

\begin{beamercolorbox}[sep=4pt,center]{palette primary}
\usebeamerfont{palette primary}primary
\end{beamercolorbox}

\begin{beamercolorbox}[sep=4pt,center]{palette secondary}
\usebeamerfont{palette secondary}secondary
\end{beamercolorbox}

\begin{beamercolorbox}[sep=4pt,center]{palette tertiary}
\usebeamerfont{palette tertiary}tertiary
\end{beamercolorbox}

\begin{beamercolorbox}[sep=4pt,center]{palette quaternary}
\usebeamerfont{palette quaternary}quaternary
\end{beamercolorbox}

\end{column}
\begin{column}{.5\textwidth}

\centering
\verb|palette sidebar|
\vspace{1ex}

\begin{beamercolorbox}[sep=4pt,center]{palette sidebar primary}
\usebeamerfont{palette sidebar primary}primary
\end{beamercolorbox}

\begin{beamercolorbox}[sep=4pt,center]{palette sidebar secondary}
\usebeamerfont{palette sidebar secondary}secondary
\end{beamercolorbox}

\begin{beamercolorbox}[sep=4pt,center]{palette sidebar tertiary}
\usebeamerfont{palette sidebar tertiary}tertiary
\end{beamercolorbox}

\begin{beamercolorbox}[sep=4pt,center]{palette sidebar quaternary}
\usebeamerfont{palette sidebar quaternary}quaternary
\end{beamercolorbox}

\end{column}
\end{columns}

\vspace{2ex}
For example, \verb|palette secondary| is defined using
\begin{verbatim}
bg=ufl blue!80!white,fg=white
\end{verbatim}

\end{frame}


\begin{frame}{Frame element colors}

The symbol colors for itemized and enumerated lists derive from the UF Bright Color Palette and progress
\begin{itemize}
\item from green,
\begin{itemize}
\item through violet,
\begin{itemize}
\item to red.
\end{itemize}
\end{itemize}
\end{itemize}
\alert{Alerted text is rendered in yellow.}

\vfill\begin{remark}
Block and caption titles are rendered in UF orange and Cantarell extra bold.
\end{remark}

\vfill
Hyperlinks and buttons are rendered in muted UF blue.
See \hyperlink{sec:customization}{Customization} and \hyperlink{slide:acknowledgments}{Acknowledgments} for examples.

\end{frame}


\section{Customization}
\label{sec:customization}


\begin{frame}[fragile]{Manually assigning colors}

To again showcase colors from the Bright Color Palette, type
\begin{verbatim}
{\color{ufl violet} violet}
\end{verbatim}
(with brackets) to render {\color{ufl violet} violet}, or set
\begin{verbatim}
\color{ufl green}
\end{verbatim}
\color{ufl green}
(without brackets) to color all text in a slide green.

\color{black}\vfill
Names for the Neutral and Preeminence Color Palettes were obtained with help from the website \href{https://www.color-blindness.com/color-name-hue/}{Color Name \& Hue}.

\center\href{http://identity.ufl.edu/color/}{\beamerbutton{Visit the Brand website}}

\end{frame}


\begin{frame}[fragile]{Changing the theme}

You can change the default colors by using setbeamercolor in the preamble, as i did:
\begin{verbatim}
\setbeamercolor*{date}{fg=ufl blue}
\end{verbatim}
\ldots or by editing the color theme file {\tt beamercolorthemeufl.sty}.
Please \hyperlink{http://identity.ufl.edu/support/contact/}{contact the Brands office} about using the derivative theme, e.g.\ one with the colors themselves altered.

\end{frame}


\section{Thanks}


\begin{frame}{Acknowledgments}
\label{slide:acknowledgments}

I relied heavily on the following resources:
\begin{itemize}
\item the \href{http://texdoc.net/texmf-dist/doc/latex/beamer/doc/beameruserguide.pdf}{Beamer class users guide}
\item Thierry Masson's Beamer \href{http://www.cpt.univ-mrs.fr/~masson/latex/Beamer-appearance-cheat-sheet.pdf}{cheat sheet}
\item University of Florida Brand (\url{http://identity.ufl.edu/})
\begin{itemize}
\item the \href{http://identity.ufl.edu/color/}{color palettes} defined in {\tt beamercolorthemeufl.sty}
\item the \href{http://identity.ufl.edu/typography/}{typefaces} approximated in {\tt beamerfontthemeufl.sty}
\item presentation tools at the \href{http://identity.ufl.edu/our-brand/}{Branding Guidelines}
\end{itemize}
\end{itemize}

\vfill
If you have suggestions for improvement, please let me know!

\center\href{https://github.com/corybrunson/beamerthemeufl/issues}{\beamerbutton{Open an issue on GitHub}}

\end{frame}


\end{document}
